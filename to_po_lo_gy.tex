\documentclass[dvipdfmx,a4paper,11pt]{jsarticle}


% 数式
\usepackage{amsmath,amsfonts}
\usepackage{bm}
\usepackage{amsthm}
\usepackage{amssymb}
\usepackage{tcolorbox}
% 画像
% \usepackage[dvipdfmx]{graphicx,color}

\usepackage[%
dvipdfmx, %欧文ではコメントアウトする
setpagesize=false,
bookmarks=true,
bookmarksdepth=tocdepth,
bookmarksnumbered=true,
colorlinks=true,
linkcolor=blue,
citecolor=blue,
urlcolor=blue,
pdftitle={},
pdfsubject={},
pdfauthor={},
pdfkeywords={}
]{hyperref}

\usepackage{pxjahyper}
\usepackage{tikz}
\usetikzlibrary{intersections,calc,arrows.meta}


\begin{document}
\theoremstyle{plain}
\newtheorem{thm}{Theorem}[section]
\newtheorem{lem}[thm]{Lemma}
\newtheorem{prop}[thm]{Proposition}
\newtheorem{cor}[thm]{Corollary}
\newtheorem{conj}[thm]{Conjecture}

\theoremstyle{definition}
\newtheorem{ass}[thm]{Assumption}
\newtheorem{dfn}[thm]{Definition}

\theoremstyle{remark}
\newtheorem{rem}[thm]{Remark}

\theoremstyle{plain}
\newtheorem*{thm*}{Theorem}
\newtheorem*{lem*}{Lemma}
\newtheorem*{prop*}{Proposition}
\newtheorem*{cor*}{Corollary}
\newtheorem*{conj*}{Conjecture}

\theoremstyle{definition}
\newtheorem*{ass*}{Assumption}
\newtheorem*{dfn*}{Definition}

\theoremstyle{remark}
\newtheorem*{Proof}{Proof}

\numberwithin{equation}{section}

\theoremstyle{remark}
\newtheorem*{rem*}{Remark}

\title{位相幾何}
\date{}
\author{Fefr}
\maketitle
\tableofcontents

%--------------------本文--------------------%

\section{位相空間の(コ)ホモロジー}

\subsection{圏と関手}

圏の定義は略。\\
圏の例をすこしあげる。
\begin{tcolorbox}[title = 例1]
  位相空間$X$から位相空間$Y$への写像の族$f_{i}:X\to Y$に対し、写像$F:X\times [0,1] \to Y$を
  \begin{equation*}
    F(x,t)=f_{i}(x)\qquad (x\in X,t \in [0,1])
  \end{equation*}
  で定義するとき、$F$が連続ならば写像族$\{f_i\}$を$f_0$から$f_1$への\textgt{ホモトピー}(homotopy)という。\\
  連続写像$f,f':X\to Y$
\end{tcolorbox}


\end{document}
